\chapter{Аналитическая часть}
В данном разделе рассматриваются алгоритмы нахождения расстояний Левенштейна и Дамерау-Левенштейна.

\section{Описание алгоритмов}
Растояние Левенштейна\cite{levenshtein} равно минимальному числу односимвольных операций преобразования (удаление, вставка, замена), необходимых для получения одной строки из другой.

\subsection{Расстояние Левенштейна}
Заданы две строки $S_1$ и $S_2$ над некоторым алфавитом длиной $M$ и $N$ соответственно, тогда расстояние Левенштейна $d(S_1, S_2)$ можно найти по рекуррентной формуле:
\begin{eqnarray}
D(i,j) = \begin{cases}
0,\hfill i=0, j=0 \\
i,\hfill j=0, i>0 \\
j,\hfill i=0, j>0 \\
min\{\\
\quad D(i,j-1) + 1,\\
\quad D(i-1,j) + 1, \hfill j>0,i>0\\
\quad D(i-1,j-1) + m(S_1[i], S_2[j])\\
\} 
\end{cases}
\end{eqnarray}
\noindentгде $m(a,b)$ равна нулю, если $a = b$ и единице в противном случае\newline
\noindent$min(a,b,c)$ возвращает наименьший из аргументов. 
\subsection{Расстояние Дамерау-Левенштейна}
В Алгоритме нахождения Дамерау-Левенштейна\cite{levenshtein} добавляется операция перестановки символов. В таком случае расстояние Дамерау-Левенштейна $d(S_1, S_2)$ может быть найдено по формуле:
\begin{eqnarray}
d(i,j) = \begin{cases}
max(i,j), \hfill min(i,j) = 0\\
min \begin{cases}
d(i-1, j) + 1\\
d(i, j-1) + 1\\
d(i-1,j-1) +1_{(a_i \neq b_i)}\\
d(i-2,j-2) + 1
\end{cases},  \hfill i,j>1, a_i=b_{j-1}, a_{i-1} = b_j\\
min \begin{cases}
	d(i-1, j) + 1\\
	d(i, j-1) + 1\\
	d(i-1, j-1) + 1_{(a_i \neq b_i)}
\end{cases}, \hfill иначе
\end{cases}
\end{eqnarray}

\section*{Вывод}
В разделе были рассмотрены два алгоритма нахождения расстояний между строками: Левенштейна и Дамерау-Левенштейна.