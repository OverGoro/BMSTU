\chapter{Исследователькая часть}
\section{Технические характеристики}
Характеристики используемого оборудования:
\begin{itemize}
\item Операционная система - Ubuntu 24.04.1 LTS \cite{ubuntu} Linux X86\_64 \cite{linux};
\item Оперативная память - 16 ГБ;
\item Процессор - Intel® Core™ i5-13420H 2.1GHz \cite{intel}.
\end{itemize}
\section{Описание используемых типов данных}
Используемые типы данных
\begin{itemize}
\item строка - последовательность символов типа str;
\item длина строки - целое число типа int;
\item матрица - двумерный массив типа int.
\end{itemize}
\section{Оценка памяти}
Рекурсивный алгоритм нахождения расстояния Левенштейна не использует структур для хранения промежуточных результатов, но каждый вызов функции обрабатывает только фрагмент строк, после чего происходит повторный вызов функции. В худшем случае глубина рекурсии будет равна:
\begin{equation}
len(S_1) + len(S_2)
\end{equation} 
При этом на каждом вызове используются локальные переменные: 2 типа str. В таком случае максимальное количество используемой паямти будет равно:
\begin{equation}
(len(S_1) + len(S_2)) \cdot (2 \cdot sizeof(str)) 
\end{equation}
где $sizeof$ - функция вычисления размера параметра.

Алгоритм нахождения расстояния Левенштейна с использованием дианмического программирования задействует в себе матрицу размером $len(S_1) + 1 \times len(S_2) + 1$ типа int. В матрице хранятся промежуточные результаты вычислений (Расстояние Левенштейна для каждой подстроки). Кроме матрицы в алгоритме используются 2 переменные типа int и 2 перемнные типа str. Тогда количество используемой памяти будет равно:
\begin{equation}
(len(S_1) + len(S_2)) \cdot sizeof(int) + 2 \cdot sizeof(int) + 2 \cdot sizeof(str).
\end{equation}

Динамический алгоритм нахождения расстояния Дамерау-Левенштейна аналогичен динамическому для нахождения расстояния Левенштейна, иных переменных в нум не задействуется, поэтому объем используемый паямти так же равен:
\begin{equation}
(len(S_1) + len(S_2)) \cdot sizeof(int) + 2 \cdot sizeof(int) + 2 \cdot sizeof(str).
\end{equation}

\section{Время выполнения алгоритмов}
Результаты замеров времени работы алгоритмов приведены в таблице \ref{tbl:time_mes}. Замеры проводились на строках одинаковой длины и усреднялись для каждого набора входных данных одной длины. Каждое значение получено с помощью взятия среднего из 100 измерений.

\begin{table}[h]
    \begin{center}
        \begin{threeparttable}
        \captionsetup{justification=raggedright,singlelinecheck=off}
        \caption{Время работы алгоритмов (в микросекундах)}
        \label{tbl:time_mes}
        \begin{tabular}{|c|c|c|c|c|}
            \hline
                        Длина строк(символов) & Лев.(рек.) & Лев.(мат.) & Дам.-Лев.(мат.)\\
                        \hline
                        1 & 0.61 & 0.93 & 0.90\\
                        \hline
                        2 & 2.32 & 1.55 & 1.62\\
                        \hline
                        3 & 10.05 & 2.50 & 2.65\\
                        \hline
                        4 & 48.82 & 3.60 & 4.05\\
                        \hline
                        5 & 229.41 & 5.12 & 5.72\\
                        \hline
                        6 & 1125.48 & 6.78 & 7.82\\
                        \hline
                        7 & 6229.95 & 8.96 & 10.54\\
                        \hline
                        8 & 30163.73 & 11.10 & 12.95\\
                        \hline
		\end{tabular}
	\end{threeparttable}
	\end{center}
\end{table}

На рисунках \ref{img/graph_all.eps} - \ref{img/graph_matrix.eps} представлены графики зависимости времни выполнения алгоритмов от длины входных слов.
\imggraph{graph_all.eps}{Сравнение всех алгоритмов по времени}
\imggraph{graph_matrix.eps}{Сравнение динамических алгоритмов по времени}

Наиболее эффективными являются динамические алгоритмы (использующие матрицу), так как в рекурсивном алгоритме большое количество повторных расчетов.

\section*{Вывод}
Рекурсивный алгоритм работает на несколько порядков медленне, чем алгоритм, использующий динамический подход. Динамические алгоритм вычисления расстояния Левенштейна и Дамерау-Левенштейна мало отличаются по времени работы. Оценка используемой памяти показала, что рекурсивный алгоритм требует меньше памяти, нежели алгоритмы с использованием матриц. Расход памяти у динамического алгоритма вычисления расстояния Дамерау-Левенштейна равен расходу памяти у динамического алгоритма вычисления расстояния Левенштейна несмотря на добавление новой операции.
