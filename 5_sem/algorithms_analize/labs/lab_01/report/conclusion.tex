\chapter*{Заключение}
\addcontentsline{toc}{chapter}{Заключение}
Эксперементально подтверждено различие временной эффективности рекурсивной и динамической реализации выбранных алгоритмов нахождения расстояния между строками при помощи разработанного программного обеспечения на данных, полученных замерами процессорного времени выполнения алгоритмов.

На полученных результатах можно сделать вывод, что матричная реализация значительно выигрывает в скорости при увеличении длины строк, но проигрывает по количеству используемой памяти.

В ходе выполнения лабораторной работы были решены задачи:
\begin{itemize}
\item реализованы алгоритмы нахождения расстояний Левенштейна (матричный и рекурсивный) и Дамерау-Левенштейна (матричный);
\item реализации проанализированы по затрачиваемым ресурсам (памяти и процессорного времени выполнения);
\item полученные результаты описаны и обоснованы.
\end{itemize}

