\chapter*{Введение}
\addcontentsline{toc}{chapter}{Введение}
	Расстояние Левенштейна и Дамерау-Левенштейна представляют минимальное количество односимвольных операций (вставка, удаление и  замена символов), необходимых для преобразования одной последовательности символов в другую. Используются для исправления ошибок в слове, сравнения текстовых файлов и иных операций с символьными последовательностями.\newline
	\indent\textbf{Цель лабораторной работы} --- реализация и сравнение алгоритмов нахождения расстояния Левенштейна и Дамерау-Левенштейна. Для достижения поставленной цели необходимо выполнить следующие задачи:
\begin{itemize}
\item реализовать алгоритмы поиска расстояний (Левенштейна с рекурсией, Левенштейна и Дамерау-Левенштейна с использованием динамического программирования);
\item проанализировать затраченное процессорное время рекурсивной и матричной реализации алгоритмов на основе эксперементальных данных;
\item описать и обосновать полученные результаты.
\end{itemize}