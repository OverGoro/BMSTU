\chapter{Конструкторская часть}
В разделе будут представлены схемы алгоритмов нахождения расстояний Левенштейна и Дамерау-Левенштейна, дано описание используемх типов данных, оценки памяти, описана структура ПО.
\section{Представление алгоритмов}
Входными данными для алгоритмов являются строки $S_1$ и $S_2$, выходными данными - число, искомое расстояние.

На рисунках 2.1 - 2.3 приведены схемы реализованных алгоритмов нахождения расстояний Левенштейна и Дамерау-Левенштейна.
\imgscheme{lev_recursion.eps}{Схема рекурсивного алгоритма нахождения расстояния Левенштейна}
\imgscheme{lev_table.eps}{Схема алгоритма нахождения расстояния Левенштейна с использованием динамического программирования}
\imgscheme{lev_dam_table.eps}{Схема алгоритма нахождения расстояния Дамерау-Левенштейна с использованием динамического программирования}
\section*{Вывод}
В разделе были представлены схемы реализованных алгоритмов для нахождения расстояний Левенштейна и Дамерау-Левенштейна.

