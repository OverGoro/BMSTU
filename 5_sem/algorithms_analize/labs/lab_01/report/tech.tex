\chapter{Технологическая часть}
В разделе будут приведены требования к программному обеспечению, средства реализации, листиги кода.
\section{Требования к программному обеспечению}
Входные данные: 2 символьных строки произвольной длины.

Выходные данные: искомое расстояние для выбранного метода и матрица расстояний для методов с использованием динаического программирования.

\section{Средства реализации}
В работе был выбран язык программирования Python\cite{python}. Т.к. в нем присутствуют функции для вычисления процессорного времени в библиотеке time\cite{python-time}. Время замерялось с помощью функции process\_time().

\section{Реализация алгоритмов}
На листингах 3.1 - 3.4 представлены реализации алгоритмов нахождения расстояний Левенштейна и Дамерау-Левенштейна.

\lstinputlisting[style=Python, firstline=6, lastline=10, caption=Вспомогательные функции]{../code/main.py}\newpage
\lstinputlisting[style=Python, firstline=13, lastline=23, caption=Рекурсивный алгоритм нахождения расстояния Левенштейна]{../code/main.py}\newpage
\lstinputlisting[style=Python, firstline=25, lastline=40, caption=Алгоритм нахождения расстояния Левенштейна с использованием динамического программирования]{../code/main.py}\newpage
\lstinputlisting[style=Python, firstline=42, lastline=66, caption=Алгоритм нахождения расстояния Дамерау-Левенштейна с использованием динамического программирования]{../code/main.py}\newpage
\section*{Вывод}
В разделе были рассмотерны требования для программного обеспечения, используемые средства реализации, приведены листинги кода для вычисления расстояний Левенштейна (рекурсивный алгоритм и динамический алгоритм), Дамерау-Левенштейна(динамический алгоритм), и их вспомогательных функций.