\chapter{Листинги кода}
\begin{center}

\begin{lstlisting}[style={asm}, caption= sub\_1]
; Сохранение регистров DS, AX
020A:07B9  1E					push	ds
020A:07BA  50					push	ax

; Установка значения сегментного регистра DS - 0040h
; 40H - адресное пространство BIOS
020A:07BB  B8 0040				mov	ax,40h
020A:07BE  8E D8				mov	ds,ax

; Загрузка младшего байта регистра флагов в ah
020A:07C0  9F					lahf				; Load ah from flags

; Флаг DF или старший бит IOPL установлены в ds:314h?
020A:07C1  F7 06 0314 2400			test	word ptr ds:[314h],2400h	; (0040:0314=3200h)


020A:07C7  75 0C				jnz	loc_7			; Jump if not zero

; Сброс флага IF по адресу ds:314h
020A:07C9  F0> 81 26 0314 FDFF	                           lock	and	word ptr ds:[314h],0FDFFh	; (0040:0314=3200h)

020A:07D0			loc_6:

; Загрузка AH в регистр флагов
020A:07D0  9E					sahf				; Store ah into flags

; Восстановление значений регистров AX, DS
020A:07D1  58					pop	ax
020A:07D2  1F					pop	ds
020A:07D3  EB 03				jmp	short loc_8		; (07D8)

020A:07D5			loc_7:

; Запрет маскируемых прерываний с помощью cli
020A:07D5  FA					cli				; Disable interrupts
020A:07D6  EB F8				jmp	short loc_6		; (07D0)


020A:07D8			loc_8:
020A:07D8  C3					retn
\end{lstlisting}
\pagebreak

\begin{lstlisting}[style={asm}, caption = Прерывание int 8h]
; Вызов sub_1
020A:0746  E8 0070				call	sub_1			; (07B9)

; Сохранение регистров es,ds,ax,dx
020A:0746  E8 70 00				db	0E8h, 70h, 00h
020A:0749  06					push	es
020A:074A  1E					push	ds
020A:074B  50					push	ax
020A:074C  52					push	dx

; Установка значений сегментных регистров DS, ES
020A:074D  B8 0040				mov	ax,40h
020A:0750  8E D8				mov	ds,ax
020A:0752  33 C0				xor	ax,ax			; Zero register
020A:0754  8E C0				mov	es,ax

; Инкремент младшего слова счетчика тиков по адресу ds:6Ch
020A:0756  FF 06 006C				inc	word ptr ds:[6Ch]	; (0040:006C=8F66h)

; Mладшее слово счетчика тиков = 0?
020A:075A  75 04				jnz	loc_1			; Jump if not zero

; Инкремент старшего слова счетчика тиков по адресу ds:6Eh
020A:075C  FF 06 006E				inc	word ptr ds:[6Eh]	; (0040:006E=3)

; Старшее слово счетчика тиков = 24?
020A:0760			loc_1:
020A:0760  83 3E 006E 18			cmp	word ptr ds:[6Eh],18h	; (0040:006E=3)
020A:0765  75 15				jne	loc_2			; Jump if not equal

; Младшее слово счетчика тиков = 0B0h?
020A:0767  81 3E 006C 00B0			cmp	word ptr ds:[6Ch],0B0h	; (0040:006C=8F66h)
020A:076D  75 0D				jne	loc_2			; Jump if not equal

; Сброс счетчика тиков
020A:076F  A3 006E				mov	word ptr ds:[6Eh],ax	; (0040:006E=3)
020A:0772  A3 006C				mov	word ptr ds:[6Ch],ax	; (0040:006C=8F66h)

; Установка 1 по адресу ds:70h
020A:0775  C6 06 0070 01			mov	byte ptr ds:[70h],1	; (0040:0070=0)

; Установка 3 бита регистра AL
020A:077A  0C 08				or	al,8

020A:077C			loc_2:

; Сохранение регистра AX
020A:077C  50					push	ax

; Декремент счетчика времени отключения моторчика дисковода по адресу ds:40h
020A:077D  FE 0E 0040				dec	byte ptr ds:[40h]	; (0040:0040=19h)

; Счетчик времени отключения моторчика дисковода = 0?
020A:0781  75 0B				jnz	loc_3			; Jump if not zero

; Установить флаг отключения моторчика дисковода
020A:0783  80 26 003F F0			and	byte ptr ds:[3Fh],0F0h	; (0040:003F=0)

; Отправка в порт 3F2h команды отключения моторчика 0Ch
020A:0788  B0 0C				mov	al,0Ch
020A:078A  BA 03F2				mov	dx,3F2h
020A:078D  EE					out	dx,al			; port 3F2h, dsk0 contrl output

020A:078E			loc_3:

; Восстановление регистра AX
020A:078E  58					pop	ax

; Флаг PF установлен в ds:[314h]?
020A:078F  F7 06 0314 0004			test	word ptr ds:[314h],4	; (0040:0314=3200h)
020A:0795  75 0C				jnz	loc_4			; Jump if not zero

; Загрузка младшего байта регистра флагов в AH
020A:0797  9F					lahf				; Load ah from flags
020A:0798  86 E0				xchg	ah,al
020A:079A  50					push	ax

; Косвенный вызов прерывания 1Ch
020A:079B  26: FF 1E 0070			call	dword ptr es:[70h]	; (0000:0070=6ADh)
020A:07A0  EB 03				jmp	short loc_5		; (07A5)
020A:07A2  90					nop

; Вызов прерывания 1Ch
020A:07A3			loc_4:
020A:07A3  CD 1C				int	1Ch			; Timer break (call each 18.2ms)

020A:07A5			loc_5:

; Вызов sub_1
020A:07A5  E8 0011				call	sub_1			; (07B9)

; Сброс контроллера прерываний
020A:07A8  B0 20				mov	al,20h			; ' '
020A:07AA  E6 20				out	20h,al			; port 20h, 8259-1 int command
										;  al = 20h, end of interrupt
										
; Восстановление значения регистров dx, ax, ds, es
020A:07AC  5A					pop	dx
020A:07AD  58					pop	ax
020A:07AE  1F					pop	ds
020A:07AF  07					pop	es

; Переход по адресу 020A:064Ch
020A:07B0  E9 FE99				jmp	$-164h
\end{lstlisting}


%\lstinputlisting[style={asm}]{TEMP.LST}
\end{center}